%% LyX 1.5.3 created this file.  For more info, see http://www.lyx.org/.
%% Do not edit unless you really know what you are doing.
\documentclass[11pt,english,a4paper,a4paper]{article}
\usepackage[T1]{fontenc}
\usepackage[latin1]{inputenc}
\usepackage{float}
\usepackage{color}
\usepackage{amsmath}
\makeatletter

%%%%%%%%%%%%%%%%%%%%%%%%%%%%%% LyX specific LaTeX commands.
%% Bold symbol macro for standard LaTeX users
\providecommand{\boldsymbol}[1]{\mbox{\boldmath $#1$}}

%% Because html converters don't know tabularnewline
\providecommand{\tabularnewline}{\\}

%%%%%%%%%%%%%%%%%%%%%%%%%%%%%% User specified LaTeX commands.
%% LyX 1.4.5 created this file.  For more info, see http://www.lyx.org/.
%% Do not edit unless you really know what you are doing.

\usepackage{float}
\usepackage{setspace}

\makeatletter


\doublespacing

\usepackage{latexsym}
\usepackage{graphics}
\usepackage{lscape}
\usepackage{times}
\usepackage{epsfig}

\usepackage[super,sort&compress]{natbib}
\bibpunct{[}{]}{,}{n}{}{,}

%\usepackage{cite}
%\renewcommand{\@cite}[2]{#1}  %{$^{#1\if@tempswa , #2\fi}$}

\renewcommand{\@biblabel}[1]{#1.}

\makeatother

\makeatother

\usepackage{babel}
\makeatother

\begin{document}
\onecolumn
\section*{Calculating Free Energies}

We will compute the free energy associated with an electron transfer
between a Cu(I) and Cu(II) ion. How large is this free energy? 

Since we know the reference free energy, we can verify the validity
and accuracy of our simulations. We proceed in steps, using different,
but related methods:
\begin{itemize}
\item Free energy Perturbation (Zwanzig)
\item Acceptance Ratios (Bennet)
\item Thermodynamic Integration (Kirkwood)
\item Non-Equilibrium approaches (Jarzynski)
\end{itemize}

\subsection*{System setup}

First, use your favorite molecule editor to create two Copper ions and
save the file as a pdb file. Next, manipulate this file such that the
atoms are in different goups (with different group index, if you're
unsure, look for an explanation of the pdb format on-line). Also, name
the groups CU1+ and CU1+, and both atoms CU. The reason for using the
same group names is that in the Gromos force field, which we will use
here, the Lennard-Jones parameters are different for both ions, which
could complicate things later. We will not care about the precise
nature of the atomtype, but for more precise work, one should consider
chaning also the atomtypes.

Center the two ions in a cubic box of volume 4x4x4 nm$^3$. Assuming
you have named your pdb 2Cu.pdb, you can do this with editconf:
\begin{verbatim}
editconf -f 2Cu.pdb -c -box 4 4 4 -o 2Cu_in_box.pdb
\end{verbatim}
Next, we use pdb2gmx to automatically generate a topology file for
this system. Before we do that, we check whether the atom names are
what we want them to be. Open the coordinate file you have just created
with your favorite text editor. It must look like:
\begin{verbatim}
      two copper ions in a box
    2
    1CU1+    CU    1   0.000   0.000   0.000
    2CU1+    CU    2   0.000   0.000   2.000
   4.00000   4.00000   4.00000
\end{verbatim}
If it indeed does (otherwise fix it), then use pdb2gmx as follows:
\begin{verbatim}
pdb2gmx -f 2Cu_in_box.pdb
\end{verbatim}
You will get asked which force field to use. Choose GROMOS96
43a2. Have a look at the files that are created, in particular the
topol.top file, which contains the topology of this system. Try to
understand that the topology includes the charges on the two copper
ions. We will call this situation, in which there is a 1+ charge on
the first copper ion and a 2+ charge on the second copper, state
A. THe purpose of the exercise is to compute the free energy
difference between state A and state B, in which the charge of the
first copper ion is 2+ and that of the second 1+. Shortly, we will
manualy create the topolgy of the B state by changing these charges.

Because we're interested in the free energy of electron transfer in
water, we now add water automatically with the genbox command of
gromacs:
\begin{verbatim}
genbox -cp conf.gro -cs spc216.gro -p -o solvated.pdb
\end{verbatim}
The last option ensures that the topology gets automatically updated
as well. Have a look at the modified topology file. You can also view
the solvated system with your favorite pdb viewer. You can see the
water looks a bit ordered. We will perform an energy minimization to
get rid of strain first. Use the following mdp file. Use the on-line
gromacs manual on the mdp keyword to learn and understand what each
option is doing. There are many options, which we will (and can)
ignore for now.
\begin{verbatim}
title                    = Yo
cpp                      = /usr/bin/cpp
include                  = 
define                   = 


integrator               = steep
nsteps                   = 500
comm-mode                = Linear
nstcomm                  = 1
comm-grps                = 

emtol                    = 100
emstep                   = 0.01

nstxout                  = 0
nstvout                  = 0
nstfout                  = 0

nstlog                   = 500
nstenergy                = 500

nstlist                  = 5

ns_type                  = grid
pbc                      = xyz
rlist                    = 0.9

coulombtype              = PME
rcoulomb-switch          = 0
rcoulomb                 = 0.9
epsilon-r                = 1
vdw-type                 = Cut-off
rvdw-switch              = 0
rvdw                     = 0.9

DispCorr                 = EnerPres

fourierspacing           = 0.12
pme_order                = 4
ewald_rtol               = 1e-05
ewald_geometry           = 3d
epsilon_surface          = 0
optimize_fft             = no

Tcoupl                   = berendsen
tc-grps                  = System
tau_t                    = 0.1
ref_t                    = 300

Pcoupl                   = berendsen
Pcoupltype               = isotropic
tau_p                    = 0.5
compressibility          = 4.5e-5
ref_p                    = 1.0

gen_vel                  = yes
gen_temp                 = 300
gen_seed                 = 1993

\end{verbatim}
Save these settings in a file with extension .mdp.  To generate the
input for the mdrun program, we use the gromacs-pre-processor:
\begin{verbatim}
grompp -f minimize.mdp -c solvated.gro -p topol.top 
\end{verbatim}
This creates a topol.tpr. If there are errors or warnings, check them. If a topol.tpr is created, you can perform the energy minimization (a steepest descent) with the mdrun program:
\begin{verbatim}
mdrun -v
\end{verbatim}
After minimization inspect the structure. It is not essential to reach
the required convergence threshold here as we're only interested to get
rid of strain due to incorrect placement of water molecules. You can
use pdb2gmx to convert the confout.gro into a pdb file that you can
open with your favorite molecule viewer. If all looks all-right,
proceed with neutralizing the system with 3 Cl$^-$
counter-ions. Remember from the lecture on Ewald and PME that charge
neutrality is important. We use the genion program to do this
automatically. This program needs a tpr file, so we first need to
generate that file with grompp again. We use the confout.gro ({\it i.e.} the
end structure of the minimization) as the input structure file:
\begin{verbatim}
grompp -c confout.gro -f minimize.mdp -p topol.top
\end{verbatim}
then use genion:
\begin{verbatim}
genion -s topol.tpr -nname CL- -nn 3 -nq -1 -np -pname NA+ -pq 1 -o neutral.gro
\end{verbatim}
It probably is a good idea to also energy minimize the neutral.gro
structure.

Next, we perform an 100 ps equilibration simulation. No idea if this
is enough to reach thermal equilibrium, but we'll ignore this for
now. By what criteria would you consider the system equilibrated? We
will run now at a constant temperature, using so-called
weak-coupling. The .mdp file contains the following options. Again,
make sure you understand what these options are doing. And why we use
them. If not, ask!
\begin{verbatim}
title                    = Yo
cpp                      = /usr/bin/cpp
include                  = 
define                   = 
integrator               = md
tinit                    = 0
dt                       = 0.002
nsteps                   = 50000
init_step                = 0
comm-mode                = Linear
nstcomm                  = 1
nstxout                  = 500
nstvout                  = 0
nstfout                  = 0
nstcheckpoint            = 1000
nstlog                   = 500
nstenergy                = 500
nstlist                  = 5
ns_type                  = grid
pbc                      = xyz
rlist                    = 0.9
domain-decomposition     = no
coulombtype              = PME
rcoulomb-switch          = 0
rcoulomb                 = 0.9
epsilon-r                = 1
vdw-type                 = Cut-off
rvdw-switch              = 0
rvdw                     = 0.9
DispCorr                 = EnerPres
fourierspacing           = 0.12
fourier_nx               = 0
fourier_ny               = 0
fourier_nz               = 0
pme_order                = 4
ewald_rtol               = 1e-05
ewald_geometry           = 3d
epsilon_surface          = 0
optimize_fft             = no
Tcoupl                   = berendsen
tc-grps                  = System
tau_t                    = 0.1
ref_t                    = 300
Pcoupl                   = berendsen
Pcoupltype               = isotropic
tau_p                    = 0.5
compressibility          = 4.5e-5
ref_p                    = 1.0
gen_vel                  = yes
gen_temp                 = 300
gen_seed                 = 1993

\end{verbatim}
With this input file, you can create a tpr file as before, and use it
with mdrun.  

\subsection*{Free energy Perturbation}

After equilibration, we will perform a simulation of 1 ns, and use
this trajectory for FEP. Change the number of steps in the mdp file to
request a 1 ns simulation and carry out the simulations. Save the
configurations every ps. This simulation will take 10 times longer
than the equilibrium simulation.

After the simulation is done, we will use the -rerun option of MD run
to calculate $V_B-V_A$ for every frame in the trajectory. The topology
file that we use for the rerun should thus reflect the situation in
state B (i.e. the first copper ion should have a +2 charge and the
second a +1 charge). Save the state A topology that we have been using
until now under a different name, such as topol\_B.itp and make the
required modificaiton to the [ atoms ] section in that topology file.

Use the new topology of state B to create a new run input file
(tpr). Better use a different name than before, or even better, create
a new directory and perform the new calculation there. You also need
to set the output options to 1, i.e. each step, because the trajectory
contains only snapshots every 500 steps (i.e. each ps).

After creating the new tpr file, copy also the old trajectory file belonging to state A to the current location and perform the rerun:
\begin{verbatim}
mdrun -rerun traj.trr -s state_B.tpr
\end{verbatim}
You can get access to the energies stored in the edr file with
g\_energy and select the potential energy (why?). Extract also the
potential energy from the previous simulation (state A). From these
energies, calculate the free energy from the Zwanzig equation, as was
explained in the lecture:
\begin{equation}
\Delta F = -R T\ln[\langle\exp(-\frac{(V_B-V_A)}{RT})\rangle_A]
\end{equation}
What is the free energy differnce? What is the error. Why is it so
large?

The error is due to poor overlap. We have not sampled enough
configurations in which the water molecules are oriented so as to
favour the product state. 

To reduce the error, we will perform the FEP is smaller
steps. Consider the situation in which compute the free energy for
smaller steps Cu$^+$+Cu$^{2+}\rightarrow$Cu$^{1.1+}$+Cu$^{1.9+}$,
Cu$^{1.1+}$+Cu$^{1.9+}\rightarrow$Cu$^{1.2+}$+Cu$^{1.8+}$, and so
on. Of course this is not possible in reality, where there exist no
non-integer electron charges, but because free energy is a state
function, we are free to choose whatever path we like. The total free
energy is the sum of the free energies associated with these smaller
steps.

\begin{equation}
\Delta F = \sum_{i=1}^n\delta F_i
\end{equation}
with
\begin{equation} 
  \delta F_i = -RT\ln [\langle \exp[-\frac{V(Cu^{(1+i/n)+}+Cu^{(2-i/n)+})-V(Cu^{(1+(i-1)/n)+}+Cu^{(2-(i-1)/n)+})}{RT}]\rangle_{i-1}]
\end{equation}
Unless you're handy with scripting, create $n$ subdirectories (or
more) and copy the topology file of state A to each of those. Then, in
each subdirectory $i$ change the charges of the first copper ion to
$1+i/n$ and the second to $2-i/n$. Run and re-run all simulations and
compute the total $\Delta F$. Did the error get smaller? Why is this
(remember what we discussed in the lectrure about the tails of the
distributions). For convenience in the next exercise, store the files
containing $V_{i+1}-V_i$ evaluated in ensembe ({\it i.e.} trajectory)
$i$.


\subsection*{Acceptance Ratio}

The FEP approach is assymmetric, and we have seen that even if we take
very small perturbation steps, we still do not get close to the
reference free energy value of zero. To overcome this problem, we will
now use Bennet's Acceptance Ratio on our FEP data. Hopefully you have
not deleted the files yet.

As explained in the lecture, the statistical error of each FEP step is
minimized if
\begin{equation}
  \sum_{\text{sim.} B}\frac{1}{1+\exp[\beta(V_A({\bf{r}})-V_B({\bf{r}})+C)]} = \sum_{\text{sim.} A}\frac{1}{1+\exp[\beta(V_B({\bf{r}})-V_A({\bf{r}})-C]}
\end{equation}
where the sum is over the number of snapshots at which we wrote the
potential energy to the edr file (nstenergy). The optimal
free energy is that C for which the above equation is true: $\Delta F
= C$. Because C appears on both sides of the equation, we need to do
this numerically. The most straighforward, but probably not most
efficient, way to do this is to scan over C and calculate both sides
of the equation, while increasing C. We can do this first with a rough
grid, {\it i.e.} a large spacing between subsequent values of C. From
this initial scan we take the C values which give the smallest
difference between the right and left hand sides of the equation, and
increase the resolution of the grid around that value in a second
scan.

In order to use the data from the multi-step FEP exercise, we need
also the backwards FEP data. That is, starting at last point $n$, we
need to evaluate for all steps i:
\begin{equation}
\delta F_i = -RT\ln [\langle \exp[-\frac{V(Cu^{(2-(i-1)/n)+}+Cu^{(1+(i-1)/n)+})-V(Cu^{(2-i/n)+}+Cu^{(1+i/n)+})}{RT}]\rangle_i]
\end{equation}
where the sum over $i$ runs from $n$ down to 1. Now, we're only
interested in files containing $V_{i-1}-V_i$ evaluated in ensemble
$i$, which you therefore need to create and store.

Now that you have on disc the files with $V_A$ and $V_B$ for the
forward and backward FEP for each step $i$, we search for the optimal
values of $C_i$ in each step $i$. I recommend using a spreadsheet,
MatLab, or Mathematica. Alternatively, knowing how to write scripts,
and a little bit of awk, will make your life easier.

What is the Free energy estimate from Bennet Acceptance Ratio?


\subsection*{Thermodynamic Integration}

For this exercise we will make use of Gromacs' functionality to carry
out a thermo dynamic integration. As discussed in the lecture, a free
energy estimate is obtained as 
\begin{equation}
\Delta F_{AB} = \int_1^0\langle\frac{\partial V}{\partial \lambda}\rangle_{\lambda}d\lambda\approx\sum_i^{n_{\text{steps}}}\langle\frac{\partial V}{\partial \lambda}\rangle_{\lambda}\Delta\lambda
\end{equation}
Better carry out this exercise in a new subdirectory. As before, we
will use perform the calculations in ten steps. THerefore create ten
subdirectories and copy the mdp into each of
them. Add to each of the mdp files the keywords to active the
thermodynamic integration
\begin{verbatim}
free-energy              = yes
init-lambda              = 0
delta-lambda             = 0
sc-alpha                 = 0
sc-sigma                 = 0.3
\end{verbatim}
We can control the lambda value with the init-lambda keyword. Set it to 0,0.1,0.2,...1 in the corresponding subdirectories. 

With these settings Gromacs will linearly interpolate the interaction
functions between the A state and the B state. We therefore need to
inform Gromacs about the B state as well. We will simply add the
information (charges) about the B state to the topology as follows:
\begin{verbatim}
 atoms ]
;   nr    type  resnr residue  atom  cgnr charge  mass  typeB chargeB  massB
    1     CU1+      1   CU1+   CU    1    1       63.546  CU1+ 2 63.546 
    2     CU1+      2   CU2+   CU    2    2       63.546  CU1+ 1 63.546 

\end{verbatim}
Perform a simulation at fixed $\lambda$ in each subdirectory. One of
the mdrun output files is dhdl.xvg, which contains the $\frac{\partial
  V}{\partial \lambda}$ values as a function of simulation time. You
can use the g\_analyze program, the obtain the averages and error
estimates as a function of $\lambda$. After you have created a file
containing $\langle\frac{\partial V}{\partial \lambda}\rangle_\lambda$
as a function of $\lambda$ (between 0 and 1), you can also perform the
integration from 0 to 1 with g\_analyze. What is the result of the
thermodynamic integration? Is it better than FEP and BAR?

\subsection*{Jarzynski approaches}

Create a new subdirectory and copy there the mdp and topology files
that we used for the thermodynamic integration calculations. In this
exercise, we will perform very many short slow-growth
simulations. Slow growth is a continous variant of thermodynamic
integration, in which $\lambda$ evolves continously and is thus a
function of time, such that $\lambda(t_{init})=0$ and
$\lambda(t_{\text{final}})=1$. In the mdp file add the following
information (in this example we perform 20 ps of simulation)
\begin{verbatim}
dt                       = 0.002
nsteps                   = 10000

free-energy              = yes
init-lambda              = 0
delta-lambda             = 0.0001
\end{verbatim}
We will now create many subdirectories and carry out a short (few ps)
slow growth in each of these. Again, scripting will make life easy.

After these simulations are done, collect the non-reversible works
from all simulations. You can use the g\_analyze program:
\begin{verbatim}
g_analyze -f dhdl.xvg -integrate
\end{verbatim}
Note however, that the x value is time in ps, and this ranges from 0
to 20. You thus need to divide all works by 20 to get the correct
$\lambda$ interval between 0 and 1.  After you've collected the works,
you can create a histogram with your favorite data analysis
program. Try to fit a guassian to it. Use the Jarzynski equation
(exponential average) to compute the reversible work:
\begin{equation}
\exp[-\beta\Delta F]=\frac{1}{n_{\text{sim.}}}\sum_{n_{\text{sim.}}}\exp[-\beta\int_{t_i}^{t_f}\frac{\partial V}{\partial\lambda}\frac{d\lambda}{dt}dt]
\end{equation}

 How many slow
growth steps are needed to obtain an accurate free energy estimate?


\subsection*{Conclusion}

We have tried out different approaches for computing free
energy. Clearly, the assymmetric Free energy perturbation approach
should not be used, but when used in combination with the symmetric
Bennet Acceptance Ratio approach it can deliver accurate
estimates. Thermodynamic integration is also symmetric, and when the
$\lambda$-spacing is small enough to facilitate overlap between
subsequent $\lambda$ ensembles, can provide accurate free energies as
well. The first three methods assume that the changes in the
Hamiltonian are slow enough that the system is always in
equilibrium. Hence these approaches are referred to as ``equilibrium
methods''. However, due to the limitation on the sampling time, this
is hardly the case in practice. The Jarzynski method, in contrast is
defined for systems out of equilibrium. Although this may sound as a
good thing given our limitations, very many simulations are needed to
yield an accurate free energy due to the averaging. In the end, one
therefore will not save on CPU cycles. Which method to choose from
depends, as always, on the problem at hand, and it is up to you to
decide which one you'll use. 

In this tutorial we focussed on changing charges. The methods are
general, and we could have also changed bonded interactions. However,
for conformational changes, other methods, based on Umbrella Sampling
are mostly more suitable. Umbrella Sampling will be topic of a
folow-up tutorial


\end{document}
